% Fakesection 承诺书

\newcommand{\NumberMilitaryMathModeling}{B028819084}
\newcommand{\NumberNJUSTMathModeling}{114}
\newcommand{\NumberProblem}{B}
\newcommand{\MemberOne}{}
\newcommand{\MemberTwo}{}
\newcommand{\MemberThree}{}
\newcommand{\MemberTeacher}{}
\newcommand{\MembersUniversity}{}
\newcommand{\MembersCollege}{}
\newcommand{\Year}{2019}
\newcommand{\Month}{5}
\newcommand{\Day}{4}

\newif\ifMilitaryMathModeling\MilitaryMathModelingfalse
\newif\ifNJUSTMathModeling\NJUSTMathModelingfalse
\newif\ifNoAppendix\NoAppendixfalse

\newcounter{NumberSessionMilitaryMathModeling}
\setcounter{NumberSessionMilitaryMathModeling}{\Year}
\addtocounter{NumberSessionMilitaryMathModeling}{-2016}
\ifnum\strcmp{\jobname}{\NumberMilitaryMathModeling}=0
\MilitaryMathModelingtrue
\fi
\ifnum\strcmp{\jobname}{\NumberNJUSTMathModeling}=0
\NJUSTMathModelingtrue
\fi
\ifnum\strcmp{\jobname}{NJUST}=0
\NJUSTMathModelingtrue
\NoAppendixtrue
\fi
\ifnum\strcmp{\jobname}{main}=0
\MilitaryMathModelingtrue
\NJUSTMathModelingtrue
\fi

% Fakesubsection 军事数学建模

\ifMilitaryMathModeling

\thispagestyle{empty}
\newgeometry{left=2.5cm,right=2.5cm,top=2cm,bottom=2cm}

\begin{center}
\zihao{3}

\textbf{第\chinese{NumberSessionMilitaryMathModeling}届全国大学生军事数学建模竞赛学术道德与保密承诺书}
\end{center}

\vspace{1em}

\begin{spacing}{2.0}
\fangsong
\zihao{4}

我们仔细阅读了第\chinese{NumberSessionMilitaryMathModeling}届全国大学生军事数学建模竞赛的竞赛章程。

我们完全清楚,在竞赛开始后参赛队员不能以任何方式(包括电话、电子邮件、网上咨询等)与队外的任何人(包括指导老师)研究、讨论与赛题有关的问题。

我们完全清楚,必须遵守保密规定,不能在互联网等非保密环境和设备上拷贝、张贴、上传、讨论竞赛题目和答卷,不能向任何队外人员传送竞赛题目和答卷。我们提交的竞赛论文等成果涉密等级不超过秘密。

我们知道,抄袭别人的成果是违反竞赛规则的,如果引用别人的成果或其他公开的资料(包括网上查到的资料),必须按照规定的参考文献的表述方式在正文引用处和参考文献中明确列出。

我们郑重承诺,严格遵守竞赛规则,以保证竞赛的公正、公平性。如有违反竞赛规则的行为,我们将受到严肃处理。

我们参赛的题号是(从A、B、C、D中选择一项):\NumberProblem

我们的参赛队号为:\NumberMilitaryMathModeling

所属单位(填写全称):\MembersUniversity

参赛队员:1.\underline{\makebox[8\ccwd][c]{\MemberOne}}2.\underline{\makebox[8\ccwd][c]{\MemberTwo}}3.\underline{\makebox[8\ccwd][c]{\MemberThree}}

指导老师:\underline{\makebox[20\ccwd][c]{\MemberTeacher}}
\end{spacing}

\begin{flushright}
\fangsong
\zihao{4}

日期:\Year 年\Month 月\Day 日
\end{flushright}

\restoregeometry
\setcounter{page}{1}

\fi

% Fakesubsection 南京理工大学数学建模

\ifNJUSTMathModeling

\thispagestyle{empty}
\newgeometry{left=2.5cm,right=2.5cm,top=3cm}

\begin{center}
\textbf{\zihao{4}\Year 南京理工大学大学生数学建模竞赛}

\vspace{2em}

\textbf{\zihao{3}承~诺~书}

\vspace{1em}
\end{center}

{\zihao{-4}

我们仔细阅读了中国大学生数学建模竞赛的竞赛规则.

我们完全明白,在竞赛开始后参赛队员不能以任何方式(包括电话、电子邮件、网上咨询等)与队外的任何人(包括指导教师)研究、讨论与赛题有关的问题。

我们知道,抄袭别人的成果是违反竞赛规则的, 如果引用别人的成果或其他公开的资料(包括网上查到的资料),必须按照规定的参考文献的表述方式在正文引用处和参考文献中明确列出。

我们郑重承诺,严格遵守竞赛规则,以保证竞赛的公正、公平性。如有违反竞赛规则的行为,我们将受到严肃处理。

我们授权全国大学生数学建模竞赛组委会,可将我们的论文以任何形式进行公开展示(包括进行网上公示,在书籍、期刊和其他媒体进行正式或非正式发表等)。

我们参赛选择的题号是(从A/B中选择一项填写):\underline{\makebox[12\ccwd][c]{\NumberProblem}}

我们的参赛报名号为(报名编号):\underline{\makebox[12\ccwd][c]{\NumberNJUSTMathModeling}}

所属学院(请填写完整的全名):\underline{\makebox[20\ccwd][c]{\MembersCollege}}

参赛队员(打印并签名):1.\underline{\makebox[12\ccwd][c]{\MemberOne}}

\makebox[11.5\ccwd][c]{}2.\underline{\makebox[12\ccwd][c]{\MemberTwo}}

\makebox[11.5\ccwd][c]{}3.\underline{\makebox[12\ccwd][c]{\MemberThree}}

\begin{flushright}
日期:\underline{\makebox[4\ccwd][c]{\Year}}年\underline{\makebox[4\ccwd][c]{\Month}}月\underline{\makebox[4\ccwd][c]{\Day}}日
\end{flushright}

}

\hrulefill

{\zihao{4}评阅编号(由组委会评阅前进行编号):}

\restoregeometry
\setcounter{page}{1}

\fi

